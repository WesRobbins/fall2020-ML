\documentclass[twoside,11pt]{article}

% Any additional packages needed should be included after jmlr2e.
% Note that jmlr2e.sty includes epsfig, amssymb, natbib and graphicx,
% and defines many common macros, such as 'proof' and 'example'.
%
% It also sets the bibliographystyle to plainnat; for more information on
% natbib citation styles, see the natbib documentation, a copy of which
% is archived at http://www.jmlr.org/format/natbib.pdf

\usepackage{jmlr2e}

% Definitions of handy macros can go here

\newcommand{\dataset}{{\cal D}}
\newcommand{\fracpartial}[2]{\frac{\partial #1}{\partial  #2}}

% Heading arguments are {volume}{year}{pages}{submitted}{published}{author-full-names}

\jmlrheading{1}{2000}{1-48}{4/00}{10/00}{Marina Meil\u{a} and Michael I. Jordan}

% Short headings should be running head and authors last names

\ShortHeadings{Learning with Mixtures of Trees}{Meil\u{a} and Jordan}
\firstpageno{1}

\begin{document}

\title{Learning with Naive Bayes}

\author{\name Ben Holmgren\u{a} \email benjamin.holmgren1@student.montana.edu \\
       \addr School of Computing\\
       Montana State University\\
       Bozeman, MT, USA
       \AND
       \name Wes Robbins \email wesley.robbins@student.montana.edu \\
       \addr School of Computing\\
       Montana State University\\
       Bozeman, MT, USA
       \AND
       \name Seth Bassetti \email seth.bassetti@student.montana.edu \\
       \addr School of Computing\\
       Montana State University\\
       Bozeman, MT, USA}
\editor{Ben Holmgren, Wes Robbins, and Seth Bassetti}

\maketitle

\begin{abstract}%   <- trailing '%' for backward compatibility of .sty file
This paper describes our group's findings when using a Naive Bayesian
approach for learning on various sets of data. We provide our findings on
these data sets with and without permutation. More specifically, we show
how each data set performs using two different loss functions with some
level of statistical significance. 
\end{abstract}


\begin{keywords}
  Naive Bayes
\end{keywords}

\section{Problem Statement}
state problem and hypothesis

\section{Description of approach}

\section{Presentation of results}
include graphs and stuff

\section{discussion}

\section{summary}

\vskip 0.2in
\bibliography{sample}

\end{document}
