\documentclass[twoside,11pt]{article}

% Any additional packages needed should be included after jmlr2e.
% Note that jmlr2e.sty includes epsfig, amssymb, natbib and graphicx,
% and defines many common macros, such as 'proof' and 'example'.
%
% It also sets the bibliographystyle to plainnat; for more information on
% natbib citation styles, see the natbib documentation, a copy of which
% is archived at http://www.jmlr.org/format/natbib.pdf

\usepackage{./jmlr2e}
\usepackage{amsmath}
\usepackage{subcaption}
\usepackage{algorithmic}
\usepackage{algorithm}
\usepackage{amsmath}
\DeclareMathOperator*{\argmax}{arg\,max}
\usepackage {tikz}
\usetikzlibrary {positioning}
%\usepackage {xcolor}
\definecolor {processblue}{cmyk}{0.96,0,0,0}
% Definitions of handy macros can go here

\newcommand{\dataset}{{\cal D}}
\newcommand{\fracpartial}[2]{\frac{\partial #1}{\partial  #2}}

% Heading arguments are {volume}{year}{pages}{submitted}{published}{author-full-names}

\jmlrheading{1}{2020}{1-48}{4/00}{10/00}{Seth Bassetti, Ben Holmgren, and Wes Robbins}

% Short headings should be running head and authors last names

\ShortHeadings{An Inquiry Into Evolutionary Algorithms \& the Multi Layer Perceptron}{Bassetti, Holmgren, and Robbins}
\firstpageno{1}

\begin{document}

\title{An Inquiry Into Evolutionary Algorithms \& the Multi Layer Perceptron}

\author{\name Seth Bassetti \email seth.bassetti@student.montana.edu \\
       \addr School of Computing\\
       Montana State University\\
       Bozeman, MT, USA
	\AND
	\name Ben Holmgren \email benjamin.holmgren1@student.montana.edu \\
       \addr School of Computing\\
       Montana State University\\
       Bozeman, MT, USA
       \AND
       \name Wes Robbins \email wesley.robbins@student.montana.edu \\
       \addr School of Computing\\
       Montana State University\\
       Bozeman, MT, USA}
\editor{Seth Bassett, Ben Holmgren, and Wes Robbins}

\maketitle

\begin{abstract}%   <- trailing '%' for backward compatibility of .sty file
This paper describes our group's findings when implementing evolutionary algorithms on a feedforward multi layer
perceptron with back propagation-  comparing evolutionary training to our implementation of back propagation. 
 More specifically, we show how our model performs on a set of breast cancer data, on data pertaining
to different kinds of glass, on data related to different kinds of soybeans, predicting the age of abalone,
performance values for various computer hardware, and on forest fire data.
	The performance of our model on each respective set of data
 is evaluated in the context of both classification and regression. For datasets where
 classification was performed, performance was evaluated in terms of average cross entropy. 
	For datasets where regression was performed,
 mean squared error was the chosen metric to evaluate
 our model. In these two differing settings, we used each respective evaluation of performance as our value indicative of fitness.
	We provided hypotheses for each specific dataset in terms of their overall performance in our model.
 Namely, we hypothesized the specific performance intervals and rough convergence times for each dataset, and our
 hypotheses were generally upheld. For classification and regression data, we specified rough percentage changes in fitness from start to finish
	for
our hypotheses with respect to each evolutionary model, and decided to uphold our hypotheses if they fell within 10\% of the hypothesized
value. In terms of convergence time, we specified a predicted change in convergence time between different models 
	for a given problem, and confirmed our hypothesis
if the convergence time found experimentally was within 10\% of the predicted value.
\end{abstract}


\begin{keywords}
	Neural Network, Feedforward, Back Propagation, Multi Layer Perceptron
\end{keywords}

\section{Problem Statement}
Utilizing six datasets- each from unique and differing settings, we implemented a feedforward neural network in an attempt to provide insights 
into the performance of evolutionary algorithms when applied to this kind of neural network with respect to differing kinds of data. 
The model worked on these varying data sets with the aim
of conducting supervised learning, that is- accurately guessing the class, or roughly the underlying function value in the case of regression, 
corresponding to each entry in the data given other examples of each respective
class the model is attempting to guess. More rigorously, we utilized a multilayer perceptron 
to carry out learning on regression and categorization data. In particular, three of the datasets are used to perform regression 
(Abalone, Computer Hardware, and Forest Fires),
  and three are used to perform classification (Glass, Soybean, Breast Cancer). Each of the six datasets we use in the project have a variable 
  number of classes and either discrete or real valued attributes. As the data sets are each representative of pretty drastically different situations,
  we generated completely seperate hypotheses for each kind of data with regards to its performance in our model. We chose to evaluate our evolutionary algorithm's
  performance on each set of data in multiple ways. Namely, we focused on the convergence rate and \% difference in loss when compared to back propagation 
  (either positive or negative). 
  Fitness is measured as an average loss throughout the entire neural network as data is fed in. Convergence rate is measured in the number
  of seconds it takes for our evolutionary algorithms to begin and then terminate. For each metric, if our experimental values are within $10\%$ in either direction of the
  hypothesized values, we choose to confirm the hypothesis.
  
  For the abalone data, we hypothesized that
  generally, our model would perform relatively well in this setting, and most members of the population would start with a relatively high fitness.
  As such, we thought that working more globally towards an optimum would be favorable for evolutionary algorithms, and particle swarm optimization
  would perform especially well. Rigorously, we hypothesized roughly a $10\%$ decrease in fitness values (which corresponds to an improvement in fitness)
  when compared to back propagation for our genetic algorithm, roughly a $20\%$
  decrease for our differential evolution algorithm, and roughly an $30\%$ increase for particle swarm optimization. We anticipate that our time of convergence
  will be roughly 3 seconds for $GA$ and $DE$, and roughly $3.5$ seconds for $PSO$.

  For the computer hardware data, we thought that, as with the abalone data, we would expect attribute values to be highly correlative to the classes of hardware.
  Thus, we expected a focus on global optimums to be pertinent, and we expected $PSO$ to find the most dramatic decrease in fitness values, of roughly $15\%$.
  We expected our genetic algorithm to represent a roughly $10\%$ decrease in fitness values, and our differential evolution to be roughly the same as back propagation.
  In terms of runtime, we expected genetic algorithms to perform the fastest, then differential evolution, and finally particle swarm to perform the slowest, each of these
  with statistical significance. (At least $10\%$ slower/faster than the other algorithm specified).

  For the forest fire data, we thought that predicting classes would be a bit trickier, since here we are attempting to predict the burned area of the forest.
  Just from briefly examining the data, it seems that the area burned can range anywhere from 0 to roughly 1000 hectares. However, most area values are relatively low,
  most are under 50. We had relatively low expectations for our model on
  this particular dataset, as the attribute data doesn't seem to be particularly revealing for class values. As such, we thought that bias in the model towards elite members would
  be preferable. Differential evolution, with the largest bias towards the fittest members, we thought would be the most successful here. 
  Specifically, we anticipated a $50\%$ decrease in fitness values from differential evolution, a roughly $40\%$ decrease from genetic algorithms, and 
  a roughly $10\%$ decrease from particle swarm optimization. Each model we expect to take a bit longer to converge, as we expect genetic algorithms and differential algorithms
  to converge roughly around 8 seconds, and particle swarm optimization roughly around 10. We are more or less shooting from the hip with this hypothesis. 

  Moving on to the classification data, we presumed that our model would perform reasonably well comparatively on the glass data set, which had 7 total classes. We thought that 
  since there were a large amount of classes, it might make sense for particle swarm optimization to perform particularly well. We hypothesized a roughly 10\% decrease from $PSO$,
  a roughly $10\%$ decrease from $GA$, and a roughly $5\%$ increase from DE. In terms of convergence time, we thought $PSO$ might converge at roughly the same speed as $GA$, each at 
  around $5$ seconds. We thought that $DE$ might converge a little slower, at roughly 7 seconds.

  In terms of the small soybean data, the feature values would seem to be highly related to each of the soybean classes, and the data set only featured four classes,
  meaning that randomly guessing should have a 25\% success rate. We thought that preferring the most elite members in a population here would be preferable, since there
  are relatively few classes to worry about, and moving towards them would probably happen the quickest favoring those with the highest fitness. As such, we thought
  $DE$ would have the highest percent decrease, of roughly $15\%$, and $GA$ would follow with roughly $5\%$, and $PSO$ would bring up the rear with a $5\%$ or so increase.
  We thought that because of these somewhat clear cut distinctions, $DE$ and $GA$ would converge the fastest, each at roughly 3 seconds. $PSO$ we thought would
  converge at roughly 4 seconds. 

  Lastly, for the breast cancer data, we expected to find that our model also performed quite well. In terms of specific values, the breast cancer categorized
  only two classes (malignant and benign cancer), so random guessing would provide correct answers in theory 50\% of the time. In this kind of setting, we thought
  that the most elite members of the population would be similarly highly favored as in the soybean data. However, we also thought that normal backpropagation would
  be quite effective. Thus, we anticipated a $25\%$ reduction in
  fitness (loss) values from $DE$, and roughly $10\%$ from $GA$. We didn't think particle swarms would be particularly helpful here, and expected an increase of 
  roughly $30\%$ in fitness values from $PSO$ when compared to back propagation. 
  In terms of runtime, we also thought that $DE$ would be really fast, roughly $2$ seconds per run, with $GA$ at roughly
  3 seconds per run, and $PSO$ at roughly 6 seconds.

  Wholistically, we expect to find that the more complex of differentiation (different possible categories, huge variation in function values) a neural net sets out to do, 
  the better more globally focused algorithms like $PSO$ will do. We expect generally for $DE$ to perform better when the delineation between different classes is more clear.
  Generally, we anticipate $GA$ will be somewhere in between the two, but nearer to $DE$.

  Note that we chose to measure the performance of evolutionary algorithms with respect to the percent change in their fitness values when comparing the outputs of two
  different methods. In doing so, we're able to categorize the performance of these algorithms in how effectively they may alter a population
  to be more fit after a fixed number of generations- based most specifically just on the holistic performance of the algorithms rather than the performance of
  our model on generic data.
  


\section{Methods}
In order to test our hypotheses, we ran our model using classification on the Glass, Breast Cancer, and Small Soybean data sets, and 
we ran our model using regression on the Abalone, Forest Fire, and Machine data. In order to successfully interpret the data, all feature values
which were not originally real valued were converted to real number values. The only data set with missing attribute values was the Breast cancer
data set, which included `?' values in place of missing data. We chose to assign each of these missing attribute values with the mean value found for 
that specific attribute in the data. We acknowledge that our
solution in this case reveals potential vulnerabilities in our model, but we justify our choice by the large number of entries in the breast
cancer data set, along with the presence of a wealth of attributes to delineate only two classes. 
Otherwise, preprocessing occurred by binning categorical data and assigning bins to real numbers. Furthermore, we used z-score normalization 
for each numeric attribute value so that the values wouldn't require any special handling.
z-score normalization is defined as follows:
\begin{equation}
	z = (x-\mu)/\sigma
\end{equation}
Where a value $x$ is subtracted by the mean $\mu$ of a sample and then divided by that sample's standard deviation $\sigma$.

After all of the datasets had been preprocessed, we implemented the multi
layer perceptron, with back propagation. As this implementation is not at the heart of this experiment,
for details of the exact equations we used and our exact implementation of the MLP, see our programming assignment 3.

After the model is trained and makes its guesses, we determine model performance using 2 different loss functions- which we deem indicative of
the evolutionary fitness of that particular neural net.
These loss functions provide an effective method of determining model performance when certain features of the model are changed. Explicitly, letting
$N$ be the number of samples, $k$ be the number of classes, $t_{i,j}$ being the value $1$ if sample $i$ is in class $j$ and $0$ otherwise, and $p_{i,j}$ being the
predicted probability that sample $i$ is in class $j$, the average cross
entropy was computed as follows:
\begin{equation}
	cross entropy = -\frac{1}{N}\sum_{i=1}^{N}\sum_{j=1}^{k}t_{i,j}ln(p_{i,j})
\end{equation}

Cross entropy is helpful when evaluating the performance of a model while also keeping in mind the potential of overfitting. By incorporating punishments for a choice being more 
`surprising' in that the chances of a correct choice are relatively small, cross entropy is useful in capturing the performance of a model with greater depth than simply 
counting the number of correct solutions- which in turn provides a weariness for overfitting.
We chose to use this loss function because we thought it would be more important in guaging the efficacy of our evolution- without too
heavily overfitting data. 

For regression data, we evaluated the performance of our model using mean squared error. Mean squared error is computed by
simply finding the mean of the squares of the errors. Mathematically, letting $Y$ be the vector of $n$ observed values and $\hat{Y}$ be the vector of $n$ predicted
values, mean squared error $MSE$ is computed as:
\begin{equation}
MSE = \frac{1}{n} \sum_{i=1}^{n} (Y_i - \hat{Y_i})^2
\end{equation}

Both metrics were used as fitness values for a given neural network in the context of evolutionary algorithms. After the neural network had been initialized and evaluated
in this manner, we incorporated the Genetic algorithm, differential evolution algorithm, and particle swarm optimization. Though we won't include their pseudocode here,
as the algorithms are well known and documented, we will provide our hypertuning of specific parameters within each algorithm, and justification for our hypertuning. For all
algorithms, a population of 40 individuals was used- as this seemed a sizable enough population to experience serious, meaningful evolutionary changes without drastically
harming the runtime. Measuring timing of different algorithms was also an interesting problem for our group. We chose to focus on a fixed number of iterations for each algorithm,
and to measure the time it took on average for one iteration (generation) to proceed to the next. Certainly, other interpretations are plentiful, this was just the approach
we chose to focus on strict experimental runtime in our study.

\begin{enumerate}
\item Particle Swarm Optimization:

	Particle Swarm Optimization has hyper parameters commonly noted as $\omega$, swarmsize, $c_1, c_2$. $\omega$ 
		affects the strength of the intertia term when calculating velocity. 
		The value of $\omega$ is between the range 0 and 1. A lower $\omega$ value allows for more dynamic 
		movement of particles with in the swarm allowing for finding a solution for quickly. 
		However, by having a small interia term the alogorithm is at a higher risk of getting stuck in local optimum. 
		After tuning we found $\omega = .8$ to be the optimum value that prevented getting stuck in local optima, but 
		did not overpower the cognitive and social terms. The hyperparameters $c_1, c_2$ act as coefficents for the social 
		and cognitive terms respectively. We found the optimally tuned values of these to be dependant on the topology used 
		for the swarm. In a gBEST smarm we found it optimal values to be $c_1 = 1.2, c_2 = .6$. In this case we have $c_1$ 
		equal to twice the value to $c_2$ in order to comboat selection pressure applied by high fitness of the social term in a gBEST 
		topology. When using an lBEST topology we tuned $c_1 = .8, c_2 = .8$. The swarm size hyperparamter represents the number of particles in the swarm. 
		We tuned this value to 10 because we found it was sufficiently large, while still being small enough to keep runtime relatively low.
  
\item Differential Evolution:

	Differential Evolution is especially customizable. Namely, in our implementation, we iterate over all vectors in a population $\mathbb{P}$, and call each vector
		the targed vector, denoted $\mathbb{X}_j$. Then we also choose three other distinct vectors $\{\mathbb{X}^1, \mathbb{X}^2, \mathbb{X}^3\}\in \mathbb{P}$.
		From these vectors, we compute a trial vector $\mathbb{U}_j$ as follows:

		\begin{equation}
			\mathbb{U}_j := \mathbb{X}^1 + \beta(\mathbb{X}^2 - \mathbb{X}^3) | \beta \in [0,2]
		\end{equation}

		The above constitutes the mutation operation for differential evolution. Below, we now show the crossover operation as follows:
		\begin{equation}
		\mathbb{X}_j'=
		\begin{cases}
			\mathbb{X}_{ji} \text{ if } rand(0,1)\leq Pr \\
			\mathbb{U}_j \text{ else}
		\end{cases}
		Pr = 0.5
		\end{equation}

		Where $\mathbb{X}_j'$ denotes the crossed over $\mathbb{X}_j$ and $Pr$ is a tunable parameter for a crossover probability. We chose to stick to $0.5$ for this
		value, as we seemed to get the best performance with this setting in trials.

		We then check to see if the target vector has higher fitness than the crossed over vector. If so, we just keep the original target vector. Otherwise, we replace
		the target vector with the vector $\mathbb{X}_j'$ which came from the combination of our mutation and crossover operations. As a result, $DE$ has a tendency to
		be a bit elitist. In the notation common to the field, we chose to use the implementation denoted ``\textit{DE/curr/2/Bin}" Where we conducted DE choosing the 
		current organism while iterating through the population, with a total of 2 difference vectors in the mutation, and where we used binomial crossoveri.

\item Genetic Algorithm:
	In our genetic algorithm implementation, we iteratively choose pairs in the population randomly, comparing their fitness. Whichever member of the pair has greater
		fitness, we add to the selected population. We then conduct the crossover operation, doing so uniformly, and on another randomly designated pair, iteratively
		over the entire population. More rigorously, honoring some crossover rate $\mathbb{C}$ 
		value in $[0,1]$, we choose to do gene crossover whenever a random variable on the same interval is less than $\mathbb{C}$. This is done by randomly assigning
		two children genes (weights) corresponding to a random mix of the weights of the original pair (parents). If the random variable is not less than $\mathbb{C}$,
		we just keep the randomly chosen pair the same for the next generation of the genetic algorithm. Finally, mutation occurs uniformly based on random number generation
		within some range, to be added or subtracted to a given gene, based on a mutation rate. We found experimentally that retaining a mutation rate and a mutation amount
		of roughly $0.2$ seemed to add in the perfect amount of random genetic diversity into the population, without introducing too much randomness to outweigh
		evolutionary processes. However, for the genetic algorithm we found it optimal to use a much higher frequency of crossover. We rationalize this decision with the
		rationale that adding in crossover for selected individuals introduces genetic diversity- but only from the already determined most fit individuals. Which is extremely
		beneficial for a population in the long term.


\end{enumerate}

\section{Results}
We now provide our results when running our various implementations of genetic algorithsm on the named sets of data. We compare these results alongside other genetic algorithms,
as well as against standard back propagation of our multi layer perceptron. We conduct all evolutionary algorithms on a population of 40 individuals. Here we record the final
fitness value provided by each of our algorithms, which is an average fitness value across the entire population in that generation.

We begin by providing the performance of our model on the breast cancer categorization data. Our results with respect to runtime and fitness measured in cross entropy are as
follows:

\begin{center}
	\begin{tabular}{|c c c c c|}
		\hline
		Metric & GA & DE & PSO & Back Propagation \\ [0.5ex]
		\hline \hline
		Fitness (entropy)  & 0.118 & 0.208 & 0.205 & 0.129 \\
		\hline
		Convergence(sec) & 2.914 & 2.899 & 3.557 & 11.2 \\
		\hline
	\end{tabular}
\end{center}

Next, we provide our findings on the glass data:

\begin{center}
	\begin{tabular}{|c c c c c|}
		\hline
		Metric & GA & DE & PSO & Back Propagation \\ [0.5ex]
		\hline \hline
		Fitness (entropy)  & 1.225 & 1.729 & 1.177 &  1.1024\\
		\hline
		Convergence(sec) & 1.140 & 0.983 & 1.026 & 3.76\\
		\hline
	\end{tabular}
\end{center}

For the final set of categorical data, we investigate the small soybean data:

\begin{center}
	\begin{tabular}{|c c c c c|}
		\hline
		Metric & GA & DE & PSO & Back Propagation \\ [0.5ex]
		\hline \hline
		Fitness (entropy)  & 0.741 & 1.184 & 0.630 & 1.199 \\
		\hline
		Convergence(sec) & 0.237 & 0.242 & 0.265 & 15.89\\
		\hline
	\end{tabular}
\end{center}

For the first set of regression data, we present the abalone data, and our findings with
respect to the performance of our varying models in solving this problem.
\begin{center}
	\begin{tabular}{|c c c c c|}
		\hline
		Metric & GA & DE & PSO & Back Propagation \\ [0.5ex]
		\hline \hline
		Fitness (mean squared)  & 39.062 & 193.095 & 107.711 & 60.725 \\
		\hline
		Convergence(sec) & 15.286 & 15.501 & 15.498 & 46.69\\
		\hline
	\end{tabular}
\end{center}

Next, we present our findings on the machine data as follows:
\begin{center}
	\begin{tabular}{|c c c c c|}
		\hline
		Metric & GA & DE & PSO & Back Propagation \\ [0.5ex]
		\hline \hline
		Fitness (mean squared)  & 35844.32 & 39820.496 & 36896.332 & 26006.236 \\
		\hline
		Convergence(sec) & 0.777 & 0.796 & 0.945 & 3.71\\
		\hline
	\end{tabular}
\end{center}

Finally, we present our findings on the forest fire data- the most challenging problem present
in our collections of data.
\begin{center}
	\begin{tabular}{|c c c c c|}
		\hline
		Metric & GA & DE & PSO & Back Propagation \\ [0.5ex]
		\hline \hline
		Fitness (mean squared)  & 463.795 & 100.073 & 479.550 & 490.241 \\
		\hline
		Convergence(sec) & 1.905 & 1.909 & 1.24 & 7.38\\
		\hline
	\end{tabular}
\end{center}

\section{Discussion}
In this section, we mention our experimental findings as they pertain to our previously generated hypotheses,
and provide some commentary on possible explanations of our findings- which were quite varied.

We begin by discussing our findings on the breast cancer data. The genetic algorithm was the
strongest performer in terms of fitness, with differential evolution and particle swarm 
optimization lagging a bit behind. We significantly underestimated the performance of back propagation in this section, which outperformed both differential evolution and PSO by a sizeable
margin (roughly 31\% lower than each in fitness). However, genetic algorithms were still able to outperform back propagation. We can comfortably fail to uphold our hypotheses for both of 
these methods. Our hypothesis for the decreases brought about by switching from backpropagation
to genetic algorithms were much closer to being confirmable. We can confirm this hypothesis, asthe decrease brought about by choosing genetic algorithms sat at roughly within 5 percent of 
15\%, which was all we deemed necessary. In terms of runtime, we were outside of statistical
significance for all of our hypotheses, comfortably failing to uphold any of them.

We now discuss our findings on the glass data. We'd expected small improvements from using
PSO and GA, and small loss in performance using DE in this scenario. In practice, back propagation defeated all of our evolutionary algorithms in performance, causing us to definitely not uphold our hypotheses in htis regard. In terms of convergence time, we were much faster than
we'd hypothesized. DE was the fastest performer in this setting, but all algorithms blew our 
hypotheses out of the water.

For the final set of classification data, we discuss the soybean dataset. In this problem,
we found that our model was most effective in the context of the PSO algorithm, though
all evolutionary algorithms won out over back propagation in this setting. However, we were
incorrect in our hypothesis regarding the exact performance of each of these models with
respect to one another. We'd predicted PSO would be the worst of the three, and in practice
it was the best. All of our performance values were slightly vetter than hypothesized, as well.
In terms of runtime, We'd thought DE and GA would be the fastest, converging around 3 seconds,
with PSO converging around 4 seconds. We were way off. GA, DE, and PSO all converged
with roughly the same timing, with GA converging the fastest of any method.

Moving on to the regression data, we found in the case of the abalone data that our results
were quite varied. Our best performer was by far the genetic algorithm, and it was the only
algorithm out of the three evolutionary algorithms to outperform backpropagation. We couldn't
have been further off in our hypothesis. Though, technically outperformed our hypothesis
in our judgement of the genetic algorithm's performance, we significantly underperformed
our hypotheses with respect to DE and PSO. We'd expected PSO to significantly outstrip
backpropagation, which was quite far from our findings. However, in terms of runtime, 
all models performed quite similarly. GA was the best performer, but only very slightly.
Our hypothesis was way off in the context of explicit values, but in spirit, our hypothesis
was correct. We'd expected all models to converge similarly on the abalone data, and indeed
this is what we found, with a slight edge in performance from GA and DE over PSO.

For the machine data, we again found that the best performer was GA. However, again, our
hypotheses were dead wrong in the context of backpropagation, which outperformed all of 
our evolutionary algorithms by a quite significant margin (roughly 25\%). GA was favorable
here for runtime, with DE coming in only slighyly slower on average. This portion of the
experiment, along with a handful of others, was revealing at the real power of backpropagation
in conducting learning effectively.

Finally, for the forest fire data, we had by far the best performance, and indeed the best
performance from throughout the entire set of experiments, for DE. Here, we significantly
overshot our hypothesis, with DE improving upon backpropagation by roughly 490\%. This makes
sense on some level, since DE significantly favors elitism. In a weird dataset like the
forest fire data, this may be preferable. And indeed, from our findings, DE was vastly superior
in this kind of difficult, unusual setting for data. In the context of our other hypotheses,
GA underformed our hypothesis by a statistically significant margin, and PSO was dead on.


\section{Summary}
The takeaways from this experiment are quite varied, and numerous. We gained a strong foundation
in evolutionary algorithms, and have come away with a much better understanding of the various
performances of evolutionary algorithms of different forms. We found that, generally speaking, 
genetic algorithms seemed to be the most powerful of the evolutionary algorithms studied- 
especially on data that might be considered relatively well behaved. Particle swarm optimization was
particularly effective on data which might also be considered somewhat well behaved, but with more
complex and spread apart classes generally than the class of datasets described in the prior case.
Differential evolution proved to be especially effective in difficult, less well behaved datasets,
which came in our context from the forest fire data. Here, DE outshined all other algorithms significantly.
Another major takeaway was the impressive performance of back propagation when compared to evolutionary
algorithms. Despite being much more simple relatively speaking than all of these intricate genetic algorithms,
often backpropagation was actually the most powerful technique. Certainly, these kinds of algorithms are an
interest in future work. Specifically, our group hopes to focus even more on the convergence of these kinds of
algorithms based on the eventual decrease in improvements in fitness as generations progress. We had wanted to
study runtime in this context of convergence, but our implementation was specific to a limited number of generations
in each algorithm. We gained insight into the comparative performances of these algorithms, but convergence is another
especially interesting avenue to pursue more in the future.

\end{document}
